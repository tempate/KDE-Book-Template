%%%%%%%%%%%%%%%%%%%%%%%%%%%%%%%%%%%%% 
% Check out the accompanying book, Even Better Books with LaTeX the Agile Way in 2023, for a discussion of the template and step-by-step instructions. https://amzn.to/3HqwgXM https://leanpub.com/eBBwLtAW/
% The template was originally created by Clemens Lode, LODE Publishing (www.lode.de), on 1/1/2023. Feel free to use this template for your book project! 
% I would be happy if you included a short mention in your book in order to help others to create their own books, too ("Book template based on \textit{Even Better Books with LaTeX the Agile Way in 2023} by Clemens Lode").
% Contact me at mail@lode.de if you need help with the template or are interested in our editing and publishing services.
% And don't forget to follow us on Instagram! https://www.instagram.com/lodepublishing/ https://www.instagram.com/betterbookswithlatex/
%%%%%%%%%%%%%%%%%%%%%%%%%%%%%%%%%%%%%



% This configures the PDF/EPUB output.
%%%%%%%%%%%%%%%%%%%%%%%%%%%%%%%%%%%%% 
% Check out the accompanying book, Even Better Books with LaTeX the Agile Way in 2023, for a discussion of the template and step-by-step instructions. https://amzn.to/3HqwgXM https://leanpub.com/eBBwLtAW/
% The template was originally created by Clemens Lode, LODE Publishing (www.lode.de), on 1/1/2023. Feel free to use this template for your book project! 
% I would be happy if you included a short mention in your book in order to help others to create their own books, too ("Book template based on \textit{Even Better Books with LaTeX the Agile Way in 2023} by Clemens Lode").
% Contact me at mail@lode.de if you need help with the template or are interested in our editing and publishing services.
% And don't forget to follow us on Instagram! https://www.instagram.com/lodepublishing/ https://www.instagram.com/betterbookswithlatex/
%%%%%%%%%%%%%%%%%%%%%%%%%%%%%%%%%%%%%

% This command enables checking if XeLaTeX is used.
\usepackage{ifxetex}

\ifxetex
    % For the PDF output, load the following additional packages.
    
    % This adjusts figures to fit into the width of a page.
    \usepackage{adjustbox}
    
    % Use this for fancy lines at the beginning of chapters and the end of sections.
    \usepackage{psvectorian} 

\else
    % Ignore adjustbox commands (HTML files do not have a width).
    \newcommand{\adjustbox}[2][]{#1}

    % Ignore psvectorian lines.
    \newcommand{\psvectorian}[2][]{}
\fi

% Translate newpage and hrule commands.
\ifx\HCode\undefined
    \newcommand{\nextpage}[1][]{}
\else
    \newcommand{\nextpage}[1][]{\HCode{<mbp:pagebreak />}}
    \renewcommand{\hrule}{\HCode{<hr style="clear: both" />}}
\fi

% Sets up support for multiple languages.
%%%%%%%%%%%%%%%%%%%%%%%%%%%%%%%%%%%%%
% Check out the accompanying book, Even Better Books with LaTeX the Agile Way in 2023, for a discussion of the template and step-by-step instructions. https://amzn.to/3HqwgXM https://leanpub.com/eBBwLtAW/
% The template was originally created by Clemens Lode, LODE Publishing (www.lode.de), on 1/1/2023. Feel free to use this template for your book project!
% I would be happy if you included a short mention in your book in order to help others to create their own books, too ("Book template based on \textit{Even Better Books with LaTeX the Agile Way in 2023} by Clemens Lode").
% Contact me at mail@lode.de if you need help with the template or are interested in our editing and publishing services.
% And don't forget to follow us on Instagram! https://www.instagram.com/lodepublishing/ https://www.instagram.com/betterbookswithlatex/
%%%%%%%%%%%%%%%%%%%%%%%%%%%%%%%%%%%%%

% Activate American language (and \babelEN).
% \usepackage[american]{babel}

% Activate Spanish language (and \babelES).
\usepackage[spanish]{babel}

% Activate German language (and \babelDE).
%\usepackage[ngerman]{babel}

% Fix PDF creation.
\ifxetex
	\let\pdfstrcmp\strcmp
\fi
% Set up macros to support multiple languages:
\newcommand{\babelES}[1]{\ifnum\pdfstrcmp{\languagename}{spanish}=0 {#1}\fi}
\newcommand{\babelDE}[1]{\ifnum\pdfstrcmp{\languagename}{ngerman}=0 {#1}\fi}
\newcommand{\babelEN}[1]{\ifnum\pdfstrcmp{\languagename}{american}=0 {#1}\fi}

\addto\captionsspanish{%
  \renewcommand{\contentsname}{Índice}%
}


% Loads packages to be able to configure hyphenation.
%%%%%%%%%%%%%%%%%%%%%%%%%%%%%%%%%%%%% 
% Check out the accompanying book, Even Better Books with LaTeX the Agile Way in 2023, for a discussion of the template and step-by-step instructions. https://amzn.to/3HqwgXM https://leanpub.com/eBBwLtAW/
% The template was originally created by Clemens Lode, LODE Publishing (www.lode.de), on 1/1/2023. Feel free to use this template for your book project! 
% I would be happy if you included a short mention in your book in order to help others to create their own books, too ("Book template based on \textit{Even Better Books with LaTeX the Agile Way in 2023} by Clemens Lode").
% Contact me at mail@lode.de if you need help with the template or are interested in our editing and publishing services.
% And don't forget to follow us on Instagram! https://www.instagram.com/lodepublishing/ https://www.instagram.com/betterbookswithlatex/
%%%%%%%%%%%%%%%%%%%%%%%%%%%%%%%%%%%%%

% Use fontenc to properly hyphenate accented languages.
\usepackage[T1]{fontenc}

% Add a list of words to enforce a certain hyphenation for them.
\usepackage{hyphenat}
\hyphenation{}



% In this package, we define the dimensions of the printed book.
%%%%%%%%%%%%%%%%%%%%%%%%%%%%%%%%%%%%%
% Check out the accompanying book, Even Better Books with LaTeX the Agile Way in 2023, for a discussion of the template and step-by-step instructions. https://amzn.to/3HqwgXM https://leanpub.com/eBBwLtAW/
% The template was originally created by Clemens Lode, LODE Publishing (www.lode.de), on 1/1/2023. Feel free to use this template for your book project!
% I would be happy if you included a short mention in your book in order to help others to create their own books, too ("Book template based on \textit{Even Better Books with LaTeX the Agile Way in 2023} by Clemens Lode").
% Contact me at mail@lode.de if you need help with the template or are interested in our editing and publishing services.
% And don't forget to follow us on Instagram! https://www.instagram.com/lodepublishing/ https://www.instagram.com/betterbookswithlatex/
%%%%%%%%%%%%%%%%%%%%%%%%%%%%%%%%%%%%%

%%%%%%%%%%%%%%%%%
% Configure bleed.
%%%%%%%%%%%%%%%%%

% If your book includes images that extend beyond the usual margins, set bleed to 0.125in and activate the corresponding option in Amazon KDP.

\newcommand{\margintop}{1.65cm}
\newcommand{\marginbottom}{2.0cm}
\newcommand{\marginoutside}{1.5cm}

%%%%%%%%%%%%%%%%%
% The inner margins depend on the number of pages of your book.
%%%%%%%%%%%%%%%%%

% Note that books printed by Amazon have an upper limit of number of pages. This limits depends on your book's dimensions, whether it is a paperback or hardcover format, and whether it is printed in black and white or in color.
% See https://kdp.amazon.com/en_US/help/topic/GVBQ3CMEQW3W2VL6

% Select the margin for 24 - 150 pages by default.
\newcommand{\margininside}{1.5cm}

%%%%%%%%%%%%%%%%%
% Hardcover / Softcover Formats. Selecting one of these ensures that you can use the same PDF for hardcover and softcover books on Amazon.
%%%%%%%%%%%%%%%%%

% Select 5x8 by default.
\newcommand{\bookwidth}{5in}
\newcommand{\bookheight}{8in}

%%%%%%%%%%%%%%%%%
% Hardcover-only formats. Selecting one of these requires you to create a separate PDF for a softcover version.
%%%%%%%%%%%%%%%%%

%\newcommand{\bookwidth}{8.25in}\newcommand{\bookheight}{11in}

%%%%%%%%%%%%%%%%%
% Softcover-only formats. Selecting one of these requires you to create a separate PDF for a hardcover version.
%%%%%%%%%%%%%%%%%

%\newcommand{\bookwidth}{5in}\newcommand{\bookheight}{8in}
%\newcommand{\bookwidth}{5.25in}\newcommand{\bookheight}{8in}
%\newcommand{\bookwidth}{5.06in}\newcommand{\bookheight}{7.81in}
%\newcommand{\bookwidth}{6.69in}\newcommand{\bookheight}{9.61in}
%\newcommand{\bookwidth}{7.44in}\newcommand{\bookheight}{9.69in}
%\newcommand{\bookwidth}{7.5in}\newcommand{\bookheight}{9.25in}
%\newcommand{\bookwidth}{8in}\newcommand{\bookheight}{10in}
%\newcommand{\bookwidth}{8.5in}\newcommand{\bookheight}{11in}

%%%%%%%%%%%%%%%%%
% And now we are putting everything together to set the dimensions of the book.
%%%%%%%%%%%%%%%%%

\usepackage[
    paperwidth=\bookwidth,
    paperheight=\bookheight,
    inner=\margininside,
    outer=\marginoutside,
    top=\margintop,
    bottom=\marginbottom,
    includehead,
    includefoot,
    headheight=0pt,
    bindingoffset=6mm
]{geometry}


% This package defines all the box commands.
%%%%%%%%%%%%%%%%%%%%%%%%%%%%%%%%%%%%% 
% Check out the accompanying book, Even Better Books with LaTeX the Agile Way in 2023, for a discussion of the template and step-by-step instructions. https://amzn.to/3HqwgXM https://leanpub.com/eBBwLtAW/
% The template was originally created by Clemens Lode, LODE Publishing (www.lode.de), on 1/1/2023. Feel free to use this template for your book project! 
% I would be happy if you included a short mention in your book in order to help others to create their own books, too ("Book template based on \textit{Even Better Books with LaTeX the Agile Way in 2023} by Clemens Lode").
% Contact me at mail@lode.de if you need help with the template or are interested in our editing and publishing services.
% And don't forget to follow us on Instagram! https://www.instagram.com/lodepublishing/ https://www.instagram.com/betterbookswithlatex/
%%%%%%%%%%%%%%%%%%%%%%%%%%%%%%%%%%%%%

% Replace the "Did you know?", "Read more in...", box titles, and icons if necessary.

% Configuring the commands for the PDF output...
\ifx\HCode\undefined 

    % If you want to add a picture to the top right corner of a box, uncomment the line and upload the picture.

    \usepackage[many]{tcolorbox}
    
    \newtcolorbox{problem}[1][]{colframe = black!30,colback  = black!4,coltitle = black!20!black,title=\babelDE{\textbf{Frage}}\babelEN{\textbf{Question}}
    %\hfill\smash{\raisebox{-11pt}{\includegraphics[height=1cm]{images/speech-bubble-cloud-with-question-mark.png}}}
    , #1,}
    
    \newtcolorbox{idea}[1][]{colframe = black!30,colback  = black!5,coltitle = black!30!black,title=\babelDE{\textbf{Idee}}\babelEN{\textbf{Idea}}
    %\hfill\smash{\raisebox{-11pt}{\includegraphics[height=1cm]{images/lightbulb-idea}}}
    , #1,}

    \newtcolorbox{example}[1][]{colframe = black!20,colback  = black!0,coltitle = black!20!black,title=\babelDE{\textbf{Beispiel}}\babelEN{\textbf{Example}}
    %\hfill\smash{\raisebox{-11pt}{\includegraphics[height=1cm]{images/book-and-test-tube-with-supporter}}}
    , #1,}

    
    \newtcolorbox{biography}[2][]{colframe = black!30,colback  = black!5,coltitle = black!30!black,title=\babelDE{Biographie -- }\babelEN{Biography---}\textbf{#2}
    %\hfill\smash{\raisebox{-11pt}{\includegraphics[height=1cm]{images/identity-card}}}
    , #1,}
    
% ... and for the HTML output.
\else
	
    \newenvironment{problem}[1][]{\bfseries\HCode{<b>}}{\HCode{</b>}\par}
    
    \newenvironment{idea}[1][]{\bfseries\HCode{<b>}}{\HCode{</b>}\par}
	
    \newenvironment{example}[1][]{\hrule\par \textbf{\babelDE{Beispiel}\babelEN{Example}}\par}{\hrule\par}
    
    \newenvironment{biography}[2][]{\hrule\par\textbf{\babelDE{Biographie}\babelEN{Biography}} \emdash \textbf{#2}\par}{\hrule\par}

\fi



% Print out listings as-is (ignoring any special characters).
\usepackage{listings}




\ifx\HCode\undefined 

% This code loads the \leftbar command for the definition environment.
    \usepackage{framed}
    \newenvironment{definition}[2][]{\begin{leftbar}\textbf{\textsc{#2}}\ ·\ #1}{\end{leftbar}\vspace{-\baselineskip}}


% Create a new environment "myquotation" that indents a whole paragraph to show that it is not part of the normally flowing text.
    \renewcommand{\indent}{\begin{picture}(0,0)\put(10,-5){\makebox(0,0){\scalebox{6}{\textcolor{lightgray}{``}}}}\end{picture}\hspace*{1.0cm}\hangindent=1.15cm}
    \newenvironment{myquotation}{\indent}{}

\else
    \newenvironment{definition}[2][]{\textbf{\textsc{#2}}\ ·\ #1}

% For the HTML output for the e-book, the indentation is defined in the style.css.
    \newenvironment{myquotation}
    {\begin{quotation}}{\end{quotation}}

    
\fi



% Uncomment this command for more error and warning messages.
%%%%%%%%%%%%%%%%%%%%%%%%%%%%%%%%%%%%%% 
% Check out the accompanying book, Even Better Books with LaTeX the Agile Way in 2023, for a discussion of the template and step-by-step instructions. https://amzn.to/3HqwgXM https://leanpub.com/eBBwLtAW/
% The template was originally created by Clemens Lode, LODE Publishing (www.lode.de), on 1/1/2023. Feel free to use this template for your book project! 
% I would be happy if you included a short mention in your book in order to help others to create their own books, too ("Book template based on \textit{Even Better Books with LaTeX the Agile Way in 2023} by Clemens Lode").
% Contact me at mail@lode.de if you need help with the template or are interested in our editing and publishing services.
% And don't forget to follow us on Instagram! https://www.instagram.com/lodepublishing/ https://www.instagram.com/betterbookswithlatex/
%%%%%%%%%%%%%%%%%%%%%%%%%%%%%%%%%%%%%


% Activate warnings about outdated/invalid packages.
\RequirePackage[l2tabu, orthodox]{nag}

% Configure LaTeX to provide full error messages.
\errorcontextlines 10000

% Loading this package sets up a balanced multicol environment that can end mid-page (needs to be loaded before fonts because of imakeidx package).
%%%%%%%%%%%%%%%%%%%%%%%%%%%%%%%%%%%%% 
% Check out the accompanying book, Even Better Books with LaTeX the Agile Way in 2023, for a discussion of the template and step-by-step instructions. https://amzn.to/3HqwgXM https://leanpub.com/eBBwLtAW/
% The template was originally created by Clemens Lode, LODE Publishing (www.lode.de), on 1/1/2023. Feel free to use this template for your book project! 
% I would be happy if you included a short mention in your book in order to help others to create their own books, too ("Book template based on \textit{Even Better Books with LaTeX the Agile Way in 2023} by Clemens Lode").
% Contact me at mail@lode.de if you need help with the template or are interested in our editing and publishing services.
% And don't forget to follow us on Instagram! https://www.instagram.com/lodepublishing/ https://www.instagram.com/betterbookswithlatex/
%%%%%%%%%%%%%%%%%%%%%%%%%%%%%%%%%%%%%

% Balance the contents of two columns (as opposed to filling first the left column and then the right). This is used for the glossary.
% See https://tex.stackexchange.com/questions/241094/multicol-column-balancing-only-after-a-minimum-number-of-lines

\ifxetex
\usepackage[balancingshow]{multicol}
\usepackage{regexpatch}

\newcounter{multicolminlines}
\setcounter{multicolminlines}{1}

\makeatletter
\xpatchcmd\balance@columns
   {\ifnum\dimen@<\topskip
     \mult@info\@ne
       {Start value
          \the\dimen@  \space ->
          \the\topskip \space (corrected)}%
     \dimen@\topskip
   \fi}
   {\skip@\c@multicolminlines\baselineskip
   \advance\skip@-\baselineskip
   \advance\skip@\topskip
   \ifnum\dimen@<\skip@
     \mult@info\@ne
       {Start value
          \the\dimen@  \space ->
          \the\skip@ \space (corrected)}%
     \dimen@\skip@
   \fi
   }
   {\typeout{Success!}}{\patchFAILED}
\makeatother
\else

    \newenvironment{multicols}[2][]{}{}

\fi


% Loads the bibliography support and the bibliography files.
%%%%%%%%%%%%%%%%%%%%%%%%%%%%%%%%%%%%%
% Check out the accompanying book, Even Better Books with LaTeX the Agile Way in 2023, for a discussion of the template and step-by-step instructions. https://amzn.to/3HqwgXM https://leanpub.com/eBBwLtAW/
% The template was originally created by Clemens Lode, LODE Publishing (www.lode.de), on 1/1/2023. Feel free to use this template for your book project!
% I would be happy if you included a short mention in your book in order to help others to create their own books, too ("Book template based on \textit{Even Better Books with LaTeX the Agile Way in 2023} by Clemens Lode").
% Contact me at mail@lode.de if you need help with the template or are interested in our editing and publishing services.
% And don't forget to follow us on Instagram! https://www.instagram.com/lodepublishing/ https://www.instagram.com/betterbookswithlatex/
%%%%%%%%%%%%%%%%%%%%%%%%%%%%%%%%%%%%%

%%%%%%%%%%%%%%%%%
% Set up the bibliography.
%%%%%%%%%%%%%%%%%


\ifxetex
% See https://www.ctan.org/pkg/biblatex for documentation.
	\usepackage[indexing=cite,style=authoryear,sortlocale=de_DE,natbib=true]{biblatex}
\else
	\usepackage{natbib}
	\usepackage{usebib}

	% \citetitle does not work with natbib / pdfLaTeX -> translate into \usebibentry
	\newcommand{\citetitle}[2][]{\textit{\usebibentry{#2}{title}}}
\fi

% Load the corresponding bibliography files.
\ifxetex
	\babelEN{\addbibresource{chapters/bibliography/english.bib}}
\else
    \babelEN{\bibinput{chapters/bibliography/english}}
\fi


% Add a command to allow a preface for the bibliography (optional).
\newcommand{\bibpreface}[1]{\patchcmd{\thebibliography}{\list}{#1\list}{}{}}



% Use csquotes to correctly typeset quoted text according to the selected language.
\usepackage{csquotes}


% Write the name of a referenced section or chapter label with \nameref{label}.
\usepackage{nameref}
\newcommand{\reference}[2][]{see \nameref{#2}, Chapter~\ref{#2}}


% This package loads and configures the fonts we use.
%%%%%%%%%%%%%%%%%%%%%%%%%%%%%%%%%%%%%
% Check out the accompanying book, Even Better Books with LaTeX the Agile Way in 2023, for a discussion of the template and step-by-step instructions. https://amzn.to/3HqwgXM https://leanpub.com/eBBwLtAW/
% The template was originally created by Clemens Lode, LODE Publishing (www.lode.de), on 1/1/2023. Feel free to use this template for your book project!
% I would be happy if you included a short mention in your book in order to help others to create their own books, too ("Book template based on \textit{Even Better Books with LaTeX the Agile Way in 2023} by Clemens Lode").
% Contact me at mail@lode.de if you need help with the template or are interested in our editing and publishing services.
% And don't forget to follow us on Instagram! https://www.instagram.com/lodepublishing/ https://www.instagram.com/betterbookswithlatex/
%%%%%%%%%%%%%%%%%%%%%%%%%%%%%%%%%%%%%


% Set font size of captions to small.
\ifxetex
    \usepackage[labelfont=bf]{caption}
    \captionsetup{font=small}
\fi

% Use this shorter command for textemdash.
\newcommand{\emdash}[1][]{\textemdash}

% Set bibliography to footnote size.
\renewcommand{\bibfont}{\footnotesize}
% You can use ``same'' (same font as your document's), ``sf'', ``tt''  or ``rm'' for monospaced font. Also see https://www.ctan.org/pkg/url
\usepackage{url}
\urlstyle{same}

%-------------------------------------------
% Create hyperlinks within PDF files but do not mark them as links.
\ifxetex
    \usepackage{hyperref}[2011/02/05]
    \hypersetup{hidelinks}
\fi

% The following commands improve the font face for the PDF output (font tweaks are not available for e-books).
\ifxetex
% Prevent splitting footnotes over several pages. See https://texfaq.org/FAQ-splitfoot
    \interfootnotelinepenalty=10000

% Set the footnote font size to very small.
    \renewcommand{\footnotesize}{\scriptsize}


% Set the font size of the index to very small.
    \usepackage{imakeidx}
    \indexsetup{othercode=\footnotesize}

% Use this command if you prefer more spaces between words rather than more hyphenations at the end of a line.
    \sloppy

% To slightly tweak font spacing for aesthetics, use these commands.
    \usepackage{microtype}
    \usepackage{lmodern}

% Use Linux Libertine font. For other fonts, check out http://www.tug.dk/FontCatalogue/
    \usepackage{libertine}
\fi

% Required packages to avoid warnings when switching to setting the language to Spanish
\usepackage[T1]{fontenc}
\usepackage{lmodern}
\usepackage[scale=0.9]{tgheros}


% This initializes the TikZ system.
%%%%%%%%%%%%%%%%%%%%%%%%%%%%%%%%%%%%% 
% Check out the accompanying book, Even Better Books with LaTeX the Agile Way in 2023, for a discussion of the template and step-by-step instructions. https://amzn.to/3HqwgXM https://leanpub.com/eBBwLtAW/
% The template was originally created by Clemens Lode, LODE Publishing (www.lode.de), on 1/1/2023. Feel free to use this template for your book project! 
% I would be happy if you included a short mention in your book in order to help others to create their own books, too ("Book template based on \textit{Even Better Books with LaTeX the Agile Way in 2023} by Clemens Lode").
% Contact me at mail@lode.de if you need help with the template or are interested in our editing and publishing services.
% And don't forget to follow us on Instagram! https://www.instagram.com/lodepublishing/ https://www.instagram.com/betterbookswithlatex/
%%%%%%%%%%%%%%%%%%%%%%%%%%%%%%%%%%%%%

% Tikz configuration. For details, see the main manual at  http://www.texample.net/media/pgf/builds/pgfmanualCVS2012-11-04.pdf

\usepackage{tikz}
% The pgfplots package is incompatible with tikzexternalize.
%\usepackage{pgfplots}

% Enable the usage of floats (like figures or tables) which can be placed exactly where they are defined.
\usepackage{float}

% Load various tikz libraries; you might need only some of them (or additional ones).
\usetikzlibrary{matrix,calc,positioning,shapes.arrows,shapes.symbols,decorations.pathreplacing,patterns,shapes,backgrounds,lindenmayersystems,shadings,intersections}

% Define styles of various tikz elements.
\tikzstyle{every node}=[font=\small,node distance=40pt and 50pt,thick]
\tikzstyle{textbox} = [rounded corners, text width=60pt, minimum height=50pt,text centered,draw=black]
\tikzstyle{arrow} = [thick,->,>=latex]
\tikzstyle{largearrow} = [thick,right of=book,draw,single arrow head indent=0ex,single arrow, rotate=90,node distance=130pt,text width=160pt,text centered]
\tikzstyle{block} = [rectangle,textbox]
\tikzstyle{textarr} = [rectangle,align=center,fill=white]
\tikzstyle{print} = [draw,tape,tape bend top=none,tape bend height=15pt,textbox]
\tikzstyle{wave} = [draw,tape,tape bend height=10pt,text width=60pt, minimum height=50pt,text centered,draw=black]


% This package defines the blankpage command and configures page numbering.
%%%%%%%%%%%%%%%%%%%%%%%%%%%%%%%%%%%%%
% Check out the accompanying book, Even Better Books with LaTeX the Agile Way in 2023, for a discussion of the template and step-by-step instructions. https://amzn.to/3HqwgXM https://leanpub.com/eBBwLtAW/
% The template was originally created by Clemens Lode, LODE Publishing (www.lode.de), on 1/1/2023. Feel free to use this template for your book project!
% I would be happy if you included a short mention in your book in order to help others to create their own books, too ("Book template based on \textit{Even Better Books with LaTeX the Agile Way in 2023} by Clemens Lode").
% Contact me at mail@lode.de if you need help with the template or are interested in our editing and publishing services.
% And don't forget to follow us on Instagram! https://www.instagram.com/lodepublishing/ https://www.instagram.com/betterbookswithlatex/
%%%%%%%%%%%%%%%%%%%%%%%%%%%%%%%%%%%%%

% Check https://www.overleaf.com/learn/latex/Headers_and_footers for more details.
% See https://ftp.rrzn.uni-hannover.de/pub/mirror/tex-archive/macros/latex/contrib/fancyhdr/fancyhdr.pdf
\ifxetex
    \usepackage[headings]{fancyhdr}
\fi


% Loading this package allows the redefinition of nameref (see main.tex).
\usepackage{letltxmacro}

% This package provides the ding command for special symbols.
\usepackage{pifont}

% Loading this package adds support for numbered lists.
\usepackage{enumitem}
